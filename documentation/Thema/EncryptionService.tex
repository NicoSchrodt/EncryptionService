\documentclass{article}

\usepackage{hyperref}

\title{Entwurf einer Verschlüsselungssoftware für einfache und fortgeschrittene Verschlüsselungen}
\author{Nico Schrodt}
\date{\today}

\begin{document}
	\maketitle
	\noindent Für den Programmentwurf ist geplant eine Desktop-Anwendung zu implementieren mit der eine Vielzahl an Verschlüsselungen angewendet werden kann. Vorgesehen sind einige Grundlegende Algorithmen wie die Vigenere-Chiffre oder Caesar-Chiffre, aber auch fortgeschrittenere Algorithmen wie zum Beispiel AES. Abgesehen vom Ver- und Entschlüsseln soll auch das Analysieren von Verschiedenen Texten ermöglicht werden, bspw. der Kasiski-Test oder der Friedmann-Test. Um Zeit und Aufwand zu sparen wird nur eine rudimentäre GUI zur Bedienung gegen Ende des Projekts entwickelt.\newline  \newline
\noindent
Der Zweck für den Nutzer ist zuerst einmal die Möglichkeit verschiedene Chiffren auszuprobieren um ein Einarbeiten in das Thema zu erleichtern (Daher auch die Möglichkeit zur Analyse über verschiedene Tests). Hauptnutzen sollen aber die tatsächlich brauchbaren (sicheren) Verschlüsselungen sein, mit denen idealerweise eigene Daten ver- und entschlüsselt werden können.\newline  \newline
\noindent
Geplant ist eine reine Implementierung in Python Version 3.10.0, sowie die Verwendung von Git zur Version-Control. Für die GUI ist Qt bzw. die Python-Implementierung PyQt Version 6.2 vorgesehen. Die verwendete IDEs sind PyCharm 2021.2 und der Qt Designer.\newline
\newline
\noindent
Links:\newline
\noindent
Github: \url{https://github.com/NicoSchrodt/EncryptionService}\newline
\noindent
(Vorerst Privat)\newline
\noindent
PyQt: \url{https://riverbankcomputing.com/software/pyqt/intro}
	
\end{document}