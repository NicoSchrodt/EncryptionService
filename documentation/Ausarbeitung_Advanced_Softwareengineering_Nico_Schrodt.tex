



%Autor: Nico Schrodt
%Januar 2021 - März 2021


\documentclass[12pt]{article}

\usepackage{multicol}
\usepackage{geometry}
\usepackage{blindtext}
\usepackage{setspace}
\usepackage{hyperref}
\usepackage[headsepline=0.8pt, footsepline =0.8pt]{scrlayer-scrpage}
\usepackage{listings}
\usepackage{subcaption}
\usepackage{tabularx}
\usepackage{xurl} %Formats \url{}-entrys better
\usepackage{color, colortbl}
%\usepackage{pdfpages}
\usepackage{amssymb}

\geometry{a4paper, top=25mm, left=35mm, right=25mm, bottom=25mm, headsep=13mm, footskip=12mm, head=14.5pt}

%encoding
%--------------------------------------
\usepackage[utf8]{inputenc}
\usepackage[T1]{fontenc}
%--------------------------------------

%German-specific commands
%--------------------------------------
\usepackage[ngerman]{babel}
%--------------------------------------

%Hyphenation rules
%--------------------------------------
\usepackage{hyphenat}
%--------------------------------------

\usepackage{graphicx}
\graphicspath{ bilder/}

\newcommand{\Autor}{Nico Schrodt}

\newcommand{\Bearbeitungszeitraum}{5 + 6 Semester}
\newcommand{\Kurs}{TINF19B3}
\newcommand{\Betreuer}{Daniel Lindner}

\newcommand{\DHBWLogoDeckblatt}{\includegraphics[width=4.5cm]{Logos/dhbw-logo}}
\newcommand{\IntelDataLines}{\includegraphics[width=15cm]{Bilder/Intel8080_DataLines}}

\newcommand{\Titel}{Entwerfen und Implementieren einer Verschlüsselungssoftware}
\newcommand{\ArtArbeit}{Ausarbeitung}
\newcommand{\Abschluss}{Bachelor of Science}
\newcommand{\Studiengang}{Studiengang Informationstechnik}

\newcommand{\Ort}{Karlsruhe}

%\newcommand{\Abgabedatum}{16.02.2021}


\begin{document}
\onehalfspacing
\pagenumbering{Roman}
	\begin{titlepage}
		{\DHBWLogoDeckblatt}\\[2cm]
		\begin{center}
			\vspace*{-2cm}
			{\Huge \Titel}\\[2cm]
			{\Huge \ArtArbeit}\\[2cm]
			{\Large \Abschluss}\\[0.5cm]
			{\large \Studiengang}\\[0.5cm]
			{\large an der}\\[0.5cm]
			{\large Dualen Hochschule Baden-Württemberg Karlsruhe}\\[0.5cm]
			{\large von}\\[0.5cm]
			{\large\bfseries \Autor}\\[1cm]
			{\large Abgabedatum \today}
			\vfill
		\end{center}
		\begin{tabular}{l@{\hspace{1cm}}l}
			Bearbeitungszeitraum & \Bearbeitungszeitraum \\
			Kurs & \Kurs \\
%			Ausbildungsfirma & \Ausbildungsfirma \\
			Dozent & \Betreuer \\
		\end{tabular}
	\end{titlepage}

\newpage

\thispagestyle{empty}
\begin{center}
\Large\bfseries Erklärung
\end{center}
\medskip
\noindent
Ich versichere hiermit, dass ich meine \ArtArbeit \ mit
dem Thema: 
\begin{center}
	 \Titel \ 
\end{center}
selbstständig verfasst und keine anderen als die angegebenen Quellen und
Hilfsmittel benutzt haben. Ich versichere zudem, dass die eingereichte elektronische Fassung mit der
gedruckten Fassung übereinstimmt (falls vorhanden).

\vspace{3cm}
\noindent
\underline{\Ort, \today \hspace{9cm}}\\
%\hfill\underline{\hspace{6cm}}\\
Ort, Datum\hfill Unterschrift\hspace{4cm}

\newpage

\thispagestyle{empty}
\tableofcontents

\newpage

%\thispagestyle{empty}
\thispagestyle{plain}
\cleardoublepage
\addcontentsline{toc}{section}{\listfigurename}
\listoffigures

\addcontentsline{toc}{section}{\listtablename}
\listoftables

\addcontentsline{toc}{section}{Listings}
\lstlistoflistings

\newpage

%\thispagestyle{empty}
\thispagestyle{plain}
\cleardoublepage
\section*{Abkürzungsverzeichnis}
\addcontentsline{toc}{section}{Abkürzungsverzeichnis}

\newpage
\pagenumbering{arabic}

%% Kopf und Fusszeilen==================================================== 
\pagestyle{scrheadings} % Seite mit Headern 

% loescht voreingestellte Stile 
\clearpairofpagestyles
%\clearscrheadings 
\clearmainofpairofpagestyles
%\clearscrplain 

% %%% Kopfzeile 
% einseitig: Bei einseitigem Layout, nur folgende Zeilen verwenden !!! 
%\ohead[] {\includegraphics[height=0.5cm]{Logos/Firmenlogokopfzeile}}
\ihead[]{\leftmark} % links: Kapitel
%\chead[]{} % mitte: 

% %%% Fusszeile 
%\cfoot[]{} % mitte: 
\cfoot[\pagemark]{\pagemark} % rechts: Seitenzahl


% Angezeigte Abschnitte im Header 
\automark{section}  % Inhalt von [\rightmark]{\leftmark} 

\section{Einführung}
Dieses Kapitel befasst sich vorwiegend mit relevanten Grundlagen der Arbeit. Unter anderem wird das Ziel spezifiziert, elementare Aspekte der Arbeitsweise eines Prozessors werden erläutert und die verschiedenen Werkzeuge mit denen das Ziel realisiert wird werden aufgeführt.

\subsection{Ziel der Arbeit}
In dieser Arbeit soll ein Simulationsprogramm geschrieben werden, mit dem mehrere unterschiedliche 8-Bit Prozessoren simuliert werden können. Dazu sollen die Grundlegenden Eigenschaften in kurzen Lernprogrammen erläutert werden. Ebenfalls soll es eine interaktive Einweisung geben wie der Simulator verwendet werden kann.

\section{Clean Architecture}
Platzhalter

\subsection{Geplante Schichtenarchitektur}
Platzhalter

\subsection{Umsetzung}
Platzhalter

\subsubsection{Benutzeroberfläche}
Platzhalter

\subsubsection{Verschlüsselungsdienst}
Platzhalter

\newpage

\section{Entwurfsmuster}
Platzhalter

\newpage

\section{Programming Principles}
Platzhalter

\subsection{SOLID}

\subsection{GRASP}

\subsection{DRY}

\newpage

\section{Refactoring}
Platzhalter

\subsection{Code Smells}
Platzhalter

\subsubsection{Code Smells 1}
Platzhalter
\subsubsection{Code Smells 2}
Platzhalter
\subsubsection{Code Smells 3}
Platzhalter

\subsection{Angewendete Refactorings}
Platzhalter

\subsubsection{Refactoring 1}
Platzhalter

\subsubsection{Refactoring 2}
Platzhalter

\newpage

\section{Unit Tests}
Platzhalter

\subsection{Verwendete Unit Tests und getesteter Code}
Platzhalter

\subsection{Anwendung der ATRIP-Regeln}
Platzhalter


%\newpage
%\thispagestyle{empty}
%
%\section*{Anhang}
%\addcontentsline{toc}{section}{Anhang}


\end{document}